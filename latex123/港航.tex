\documentclass[UTF8]{ctexart}
\usepackage{fancyhdr}
\usepackage{geometry}
\usepackage{amssymb}
\usepackage[fleqn]{amsmath}
\usepackage{enumerate}
\usepackage{CJK}
\usepackage{bm}

\pagestyle{fancy}
\lhead{港航商务管理}
\chead{2016.12.13}
\rhead{复习资料}
\rfoot{By 柯摩}

\renewcommand{\headrulewidth}{2pt}
\renewcommand{\footrulewidth}{0.4pt}
\date{}
\setlength{\parindent}{2em}
\geometry{left=1.27cm,right=1.27cm,top=1.27cm,bottom=1.27cm}

\begin{document}
\thispagestyle{fancy}
\pagenumbering{Roman}
\footnotesize
\begin{enumerate}[1]
\item
\begin{CJK}{UTF8}{you}
主要的常用商务法规的调整关系(内部)是什么?
\end{CJK}
\\ \textbf{答:}  调整对象是水运经济关系,它包括水运主管部门与所属水运企业之间在宏观控制方面的纵向关系,水运企业与社会团体,经济组织之间在客、货运输生产中所发生的横向经济关系,水运企业内部的生产管理关系.
\item  \begin{CJK}{UTF8}{you} 国内货物运输保险的险别及其承保的责任范围有哪些?有哪些具体规定?\end{CJK}
\\ \textbf{答:}  (1)基本险.责任范围:
\\ ①因火灾、爆炸、雷电、冰雹、暴风、暴雨、洪水、地震、海啸、地陷、崖崩、滑坡、泥石流所造成的损失.
\\ ②由于运输工具发生碰撞,搁浅,触礁,倾覆或码头坍塌造成的损失.
\\ ③在装货卸货或转载时,因遭受不属于包装质量不善或装卸人员违反操作规程所造成的损失.
\\ ④按照国家规定或一般惯例应分摊的共同海损的费用.
\\ ⑤在发生上述灾害、事故时,因纷乱而造成货物的散失以及因施救或保护货物所支付的直接合理的费用.
\\(2)综合险.责任范围
\\ ①基本险的责任范围.
\\ ②因受震动、碰撞、挤压而造成破碎、弯曲、凹痕、折断、开裂或包装破裂致使货物散失的损失.
\\ ③液体货物因受震动、碰撞、或挤压致使所用容器(包括封口)损坏而渗漏的损失,或用液体保藏的货物因液体渗漏而造成保藏货物腐烂变质的损失.
\\ ④遭受盗窃或整件提货不着的损失.
\\ ⑤符合安全运输而遭受雨淋所致的损失.
\\ 特殊规定:托运人承运的货物,对每件价值在700元以上或按重量算每吨价值500元以上,托运人应该投保,对不投保的货物,承运人可拒绝承运;由于承运人责任,对每件价值在700元以上,按实际损失最多不超过700元计算赔偿,每吨货物价值在500元以上的,按实际损失最多不超过500元计算赔偿;对每件货物的实际损失超过700元或每吨货物的实际损失超过500元的,按保险条款规定的手续向当地保险公司索赔,再由保险公司向承运人追偿由承运人负责的赔偿部分.
\item  \begin{CJK}{UTF8}{you} 何谓运到期限?如何计算运到期限?\end{CJK}
\\ \textbf{答:} 货物运到期限是指承运人在现有的技术设备条件和组织管理工作水平下,将货物运送到一定距离所需要时间上的最大限度.计算方法为:由起运港发送时间、船舶航行时间、到达港卸船准备时间3部分组成.详见P69.
\item \begin{CJK}{UTF8}{you} $PDCA$循环是哪几个阶段?关键的阶段?\end{CJK}
\\ \textbf{答:}   ①计划阶段 ②执行阶段③检查阶段④总结和处理阶段;关键在第④阶段.
\item \begin{CJK}{UTF8}{you} 排列图、因果分析图的作用?为何要综合运用?\end{CJK}
\\ \textbf{答:} 用排列图找出影响货运质量的主要因素,用因果分析图找出影响货运质量主要因素的关键原因.综合运用它们可以更加清晰高效地找出并解决问题,从而提高货运质量.
\item \begin{CJK}{UTF8}{you} 递远递近、均衡里程运价的特点(概念)?\end{CJK}
\\ \textbf{答:} 递远递近运价是对同一级别的货物每吨公里或每吨海里的运价随运距的增加而逐渐降低,而航线每吨货物的运价随运距的延长而缓缓上升.优点是能体现运距长单位成本低的规律,更能体现长距离水运,降低运价的优惠政策.缺点是计算复杂,水运常用递远递近运价;
\\ 均衡里程运价是指同一级别的货物的每吨公里或每吨海里的运价随航线距离的延长而均衡不变,而航线每吨货物运价则随运距增加而成比例上升.优点是计算简单,缺点是不能正确反映出运输距离越长,运价越低的客观规律.
\item \begin{CJK}{UTF8}{you} 实践、计划性的船舶货物运输合同的主要条款?滞期、速遣条款在什么营运方式的合同条款里面?
\end{CJK}
\\ \textbf{答:} 1.计划:①货物名称②托运人和收货人名称③起运港和到达港,海江河联运货物应载明换装港④货物重量,按体积计费的应载明体积⑤运价及港口费率⑥违约责任⑦特约条款.
\\ 2.1实践班轮:①承运人、托运人、收货人名称②货物名称、件数、重量、体积③运输费用及其结算方式④船名、航次⑤起运港、中转港和到达港⑥货物交接地点和时间⑦装船日期⑧运到期限⑨包装方式⑩识别标志\textcircled{\footnotesize{11}}违约责任\textcircled{\footnotesize{12}} 解决争议的方法.
\\2.2实践航次:①出租人和承租人名称②货物名称、件数、重量、体积③运输费用及其结算方式④船名⑤载货重量、载货容积及其他船舶资料⑥起运港和到达港⑦货物交接地点和时间⑧受载期限⑨运输期限⑩装卸期限及计算方法\textcircled{\footnotesize{11}}滞期费率和速遣费率\textcircled{\footnotesize{12}} 包装方式\textcircled{\footnotesize{13}} 识别标志\textcircled{\footnotesize{14}}违约责任\textcircled{\footnotesize{15}}解决争议的方法
\\ 滞期费与速遣费条款在实践性航次租船合同条款中.
 \item \begin{CJK}{UTF8}{you} 水路货物运价的构成?确定平均利润率的方法有几种?常用(用的最多)的是哪一种?
\end{CJK}
\\ \textbf{答:} 由货运成本、利润、税金构成.有三种,常用按社会平均劳动成本利润确定平均利润率.
 \item \begin{CJK}{UTF8}{you} 保管费率公式中,日累进系数最高不得超过多少?
\end{CJK}\\
 \textbf{答:} 不超过300\%.
\item \begin{CJK}{UTF8}{you} 装卸费率公式中,费率调整系数在什么范围之内?
\end{CJK}\\
 \textbf{答:} 0.5\~~2.
\item \begin{CJK}{UTF8}{you} JT2010-87标准中,对港航企业的考核标准是如何规定的?

\end{CJK}
 \textbf{答:} 对于港口企业:①无重大事故②散货赔偿率小于等于万分之二③件杂货赔偿率:大连,天津,上海港小于等于万分之十,其他港口小于等于万分之十五.
\\ 对于航运企业:①无重大事故②赔偿率小于等于万分之十.
\item \begin{CJK}{UTF8}{you} 我国国内运输中,水铁联运、水水联运的运价一般比非联运运价低多少?
\end{CJK}\\
 \textbf{答:} 15\% 左右.
\item \begin{CJK}{UTF8}{you} 危险货物的运单与普通货物的运单有何不同?
\end{CJK}\\
 \textbf{答:} ①危险货物需要用红色运单填写,并在运单右上角用红色标记属于危险品,而普通货物则不需要.
②普通货物一般不需要证明文件,爆炸要有危险证明文件.③放射物品应提供放射剂证明,由科技部门提供放射剂检查证明书,而普通货物则不需要.
\item \begin{CJK}{UTF8}{you} 组合基价,综合基价是确定什么运价的基础?

\end{CJK}
 \textbf{答:} 以综合基价确定的运价为均衡里程运价,以组合基价确定的运价为递远递减运价.
\item \begin{CJK}{UTF8}{you} 填写运单应注意哪些问题?
\end{CJK}
\\ \textbf{答:}①同一运单的货物,必须是同一托运人、收货人,同以起运港、到达港.
\\ ②货物的名称要写具体品名.
\\ ③易腐、易碎、流质、危险货物与普通货物,以及性质互相抵触的两种货物,不能填在同一张运单内.
\\ ④水运法规规定,按重量和体积择大计费的货物,应同时填写货物的重量和体积,为承运人计算运费提供依据;按件数和重量托运的货物应正确填写件数和重量;只按重量托运的货物可只填重量,不填件数.
\\ ⑤托运普通货物,不得夹带危险货物、流质货物、易腐货物、贵重物品、货币、有价证券和各种凭证.
\\ ⑥托运危险货物必须用专门印制的危险货物运单.
\item \begin{CJK}{UTF8}{you} 货运质量的特性有哪些?
\end{CJK}
\\ \textbf{答:} ①安全性②及时性③经济性④完整性⑤服务性
\item \begin{CJK}{UTF8}{you} 编制月度货物运输计划要做好哪几个方面的平衡工作?
\end{CJK}
\\ \textbf{答:} ①做好货物运输量与船舶运输能力之间的平衡②做好货物装卸量与港口装卸能力之间的平衡③做好联运货物换装量与港口换装能力之间的平衡
\item \begin{CJK}{UTF8}{you} 制订船舶货物运价的步骤?
\end{CJK}
\\ \textbf{答:} 首先通过综合基价或者组合基价来确定货运基本价格,然后对货物进行分级,确定每级之间的极差从而确定货类分级及极差的,之后分航区确定计算里程,最后指定运价率表.
\item \begin{CJK}{UTF8}{you} 如何利用货运事故记录判断货损、货差责任?
\end{CJK}
\\ \textbf{答:} 结合验收结束,装船结束,卸船结束,交货结束环节的货运记录和港航内部记录,逐次计算货损件数从而推断各个货损与货差分别属于谁的责任.
\item \begin{CJK}{UTF8}{you} 水路货运运价的特点?
\end{CJK}
\\ \textbf{答:}  ①货物运价具有按不同运输距离或不同航线而别的特点.②货物运价只有销售价格一种形式③货物运价的种类繁多④ 水路货物运价是商品价格的组成部分.
\item \begin{CJK}{UTF8}{you} 货运质量的特点?特性?
\end{CJK}
\\ \textbf{答:}   ①产品的形态不同②运输只保证产品原有的数量和质量③运输生产过程与消费过程具有同一性④运输过程所处环境的复杂性
\item \begin{CJK}{UTF8}{you} 联运的作用和条件?
\end{CJK}
\\ \textbf{答:}  作用:①综合利用各种运输工具,充分发挥各种运输工具的潜力
\\ ②加快车船周转,提高运输企业的经济效益.
\\ ③提高港口、车船、库场的利用率,增加吞吐能力.
\\ ④减少货物流通费用,提高社会效益.
\\ ⑤方便了货主,促进了物资流通.
\\ 条件:①要有明确的指导思想.①要进行运输组织的改革.③要有稳定的货源、客源.④要有良好的换装条件.
\item \begin{CJK}{UTF8}{you} 根据商务法规和习惯做法,国内水运中,谁是船方代表?
\end{CJK}
\\ \textbf{答:}  船代.
\item \begin{CJK}{UTF8}{you} 货主与船方要签订装卸合同是什么运输组织形式?
\end{CJK}
\\ \textbf{答:} 航次租船.
\item \begin{CJK}{UTF8}{you} 了解产品(货运质量)质量总体波动情况是什么?
\end{CJK}
\\ \textbf{答:} 直方图.
\item \begin{CJK}{UTF8}{you} 港口库场的主要功能有哪些?
\end{CJK}
\\ \textbf{答:}  ①缓冲调节功能②实施货运作业的功能③保管货物的功能.
\item \begin{CJK}{UTF8}{you} T.Q.C与传统的质量管理相比,其科学性是什么?
\end{CJK}
\\ \textbf{答:}  将数理统计方法与全过程的质量管理相结合起来.
\item \begin{CJK}{UTF8}{you} 运价与运费的关系?
\end{CJK}
\\ \textbf{答:} 运费是运价的总体,运价是单位货物运输费用.
\item \begin{CJK}{UTF8}{you} PDCA的特点?
\end{CJK}
\\ \textbf{答:} ①大环套小环,一环扣一环,推动大循环.②管理循环每转一周,货运质量就提高一步.③PDCA循环的关键在处理这一阶段④PDCA工作循环具有科学性.
\item \begin{CJK}{UTF8}{you} T.Q.C的特点?
\end{CJK}
\\ \textbf{答:} ①TQC具有全面性,管理的质量是全面的,实行的管理是全过程的、全员的,是一种灵活运用的多种管理技术和手段的质量管理.
\\②TQC管理范围是生产的全过程.
\\③TQC必须依靠企业全体职工.
\\④TQC运用多种管理技术和管理手段的综合性管理.
\item \begin{CJK}{UTF8}{you} 港口库场管理的主要内容?
\end{CJK}
\\ \textbf{答:} ①制定切实可行的年工作计划,再根据堆存能力和堆存需要情况,编制月度、旬度及日常堆存作业计划等.
\\ ②确定库场堆存定额.
\\ ③建立入库货物的验收制度和办法,规定签发入库票据以及台帐登记的办法等.
\\ ④推行堆码标准化
\\ ⑤对库场货物的保管.
\\ ⑥规定提货手续,制定货物的出库放行办法,更新台帐内容,做好信息储存工作,为统计、查询和编制计划提供依据和信息
\\ ⑦库场安全管理
\\ ⑧其他方面
\item \begin{CJK}{UTF8}{you} 港口库场管理制度有哪些?
\end{CJK}
\\ \textbf{答:} ①包线、包区、包库(场)安全负责制度.
\\②货物承运、交付、检查负责制度
\\③货物堆码、货位管理负责制.
\\④货运检查负责制度.
\\⑤库场理货员监督装卸、装卸工组负责制度.
\\⑥人员、货物出入库场等级检查负责制度.
\\⑦库场清扫、消防、保卫负责制度.
\\⑧ 库场装卸机具、货运用具和篷布的使用、保管、检修负责制度.
\\⑨票据、交接、残祸对照和物资部门自理装卸交接负责制度.
\\ ⑩货运事故检查、分析处理负责机制.
\item \begin{CJK}{UTF8}{you} 编制库场堆存计划的目的?
\end{CJK}
\\ \textbf{答:} 主要是使进出口货物的入库需要与港口现有各类库场的堆存能力和设备条件很好地结合起来,以尽量减少由于货物入库需要与堆存能力之间的矛盾给装卸生产、运输生产所带来的不利影响.
\item \begin{CJK}{UTF8}{you} 编制库场堆存计划的要求?
\end{CJK}
\\ \textbf{答:} 充分满足入库货物 的需要,做好入库货物堆存需要与堆存能力之间的平衡,尤其是在入库货物较多的情况下,要采取切实措施,使得入库货物有合适的堆存库场,以保证装卸运输的需要.要使得出入库的线路最短,要留有一定的余地.
\item \begin{CJK}{UTF8}{you} 货运质量保证体系?
\end{CJK}
\\ \textbf{答:} 货运质量保证体系是全面质量管理深入发展的主要标志,是保证全面质量管理取得长期稳定效果和巩固、扩大成果的关键.
\item \begin{CJK}{UTF8}{you} 理货?
\end{CJK}
\\ \textbf{答:} 港航理货是指在货物运输过程中,记录船舶在港口的卸货数字、核对货物标志、检查货物残损、指导和监督货物的装卸和装舱积载、绘制积载图、办理货物交接签证手续、提供有关理货单证等业务的总称.
\item \begin{CJK}{UTF8}{you} 理货的依据?(外轮理货、内理)
\end{CJK}
\\ \textbf{答:} 外轮理货:进口:舱单;\\出口:装货单;拼箱货:装箱单;整箱货:箱数;\\内理:运单,交接清单.
\item \begin{CJK}{UTF8}{you} 什么叫工残、原残?什么叫理残?
\end{CJK}
\\ \textbf{答:} 理残:在船舶装卸过程中,对残损货物应作好记录,把原残货物的积载部位、残损程度和数量记录在现场记录和残损单上,并通知船方当班复验和签认,称为理残.
\\ 工残与原残:对于出口货物,在装船前发现的残损称为原残,在装船过程中造成的残损称为工残.对于进口货物,在船上发现残损称为原残,在卸船过程中造成的残损称为工残.
\item \begin{CJK}{UTF8}{you} 什么叫理数?理数有哪几种方法?
\end{CJK}
\\ \textbf{答:} 理数是指在船舶装卸货物过程中,记录装卸货物的钩数,点清每钩的件数,计算货物数字,亦称为计数.
\\ 理数方法有:①发筹理数②划钩理数③挂牌理数④小票理数⑤点垛理货⑥自动理货.
\item \begin{CJK}{UTF8}{you} 联合运输?
\end{CJK}
\\ \textbf{答:}联合运输是指两种或两种以上的运输方式或同一运输方式的几个运输企业,遵照同意的规章或协议,联合完成某项运输任务的一种运输组织形式,简称为联运.
\item \begin{CJK}{UTF8}{you} 水铁联合运输工作计划是如何编制的?
\end{CJK}
\\ \textbf{答:} 水铁联合运输工作计划的编制,应由所有参加联合运输的水路、铁路各有关运输切也共同负责。所有签订了联运货物运输合同(托运计划表)、共同完成某一联运任务的港务局、航(海)运局、换装站要逐一落实运力、换装能力。特别是在运力、换装能力有限的情况下,要采取一切措施保证联运货物托运计划(合同)任务的完成.
\item \begin{CJK}{UTF8}{you} 什么叫商务管理?学科性质?\end{CJK}
\\ \textbf{答:} 所谓港航商务管理,就是指水路客货运输生产、经营过程中港口与航运企业具体生产业务和经营活动所进行的管理,它是水运管理科学的重要组成部分,属于运输管理与运输经济交叉的边缘学科.港航商务管理具有应用性、交叉性、而且具有很强的实践性.
\item \begin{CJK}{UTF8}{you} 现代水路货物运输市场的特征?\end{CJK}
\\ \textbf{答:} ①进入市场制度化,规范化.②水运市场的主体公平竞争.③法律手段对运输经济的宏观控制.④全国统一的水路运输市场⑤人才合理配置.⑥逐渐健全的市场规则.
\item \begin{CJK}{UTF8}{you} 水运商务法规调整的对象是什么?含哪三种关系?\end{CJK}
\\ \textbf{答:} 对象是水运经济关系,水运主管部门与所属水运企业之间在宏观控制方面的纵向关系,水运企业与社会团体、经济组织之间在客货运输生产中所发生的横向经济关系,水运企业内部的生产管理关系.
\item \begin{CJK}{UTF8}{you}什么叫水运商务法规?\end{CJK}
\\ \textbf{答:} 水运商务法规是指国家制定或确认的,并以强力保障实施的水运行为规则.
\item \begin{CJK}{UTF8}{you}水路货物运输合同的特征有哪些?\end{CJK}
\\ \textbf{答:} ①当事人可以是法人,也可以是公民.②具有计划性.③有专门的法规调整④既有实践性合同,又有计划性合同⑤收货人不参加签订货运合同,但具有合同当事人的权利和义务.⑥属提供劳务的合同⑦联运货物运输合同有特殊规定.⑧对货物的包装有特殊要求.
\item \begin{CJK}{UTF8}{you}组织货源应考虑哪些因素?\end{CJK}
\\ \textbf{答:} ①正确的货物运价策略②较高的运输速度③较好的货物运输质量④满意的运输服务⑤有力的广告宣传⑥较好的货主关系
\item \begin{CJK}{UTF8}{you} 货源组织工作的任务有哪些?\end{CJK}
\\ \textbf{答:}①运用正当的手段,为本企业组织到足够的货源②进行货源调查,掌握货源变化的规律,为编制货运计划提供可靠的货源资料③按照计划运输的要求,组织货物均衡按时集中交运④组织合理运输
\item \begin{CJK}{UTF8}{you} 什么叫托运,承运?承运、托运应做哪些商务工作?\end{CJK}
\\ \textbf{答:} 托运是指托运人按照水运法规规定的办法,委托承运人运送货物(包裹或行李)的行为.
\\ 承运是指承运人按水运法规规定的办法,接受托运人委托,承担为其运送货物的行为.
\\ 托运过程:①向承运人提出货物运单②向承运人提交托运的货物③交付运输费用
\\ 承运过程:①审核单证②验收货物②承运④收取运费
\item \begin{CJK}{UTF8}{you} 什么是到货通知?到货通知的商务作用有哪些?\end{CJK}
\\ \textbf{答:} 到货通知是指承运人向收货人发出的货物已运达且具备提货条件的通知.
\\ 商务作用:①可以让收货人及时准备接运工具,组织直取,提高运输的经济效益.
\\ ②到货通知发出的时间是计收货物保管费的依据.
\\ ②到货通知的时间是计算货物运到期限的依据之一.
\\ ④到货通知得的时间是作为超期不提货的依据.
\\ ⑤及时发出到货通知,可改善库场的利用率.
\item \begin{CJK}{UTF8}{you} 运单的新功能有哪些?\end{CJK}
\\ \textbf{答:}  ①水路货物运输合同的证明②是承运人收到货物的收据②是收货人提货的依据之一④费用结算的凭证⑤货物运单是处理货运业务的基本依据.
\item \begin{CJK}{UTF8}{you} 何谓分运、补送?其商务处理如何??\end{CJK}
\\ \textbf{答:}  分运:在联运情况下,换装港将属于一张货物运单的货物分批运出,称为分运.
\\ 分运商务处理:换装港对整批运进且需分批转运出去的联运货物,按每一批实际分运的数量填制分运运单.P77
\\ 补送: 是对一张运单的未装、漏装部分货物进行的补充发送.
\\ 补送商务处理:补送要填制补送运单;补送运单要注明原运单号码、原船名航次、原收货人、原发件数、重量等,以便于到达港凭以办理交付;交付补充货物时,若承运方在交付该票货物的基本部分时已出具承认短少货物的货运记录,应同时向收货人收回记录.
\item \begin{CJK}{UTF8}{you} 制定船舶货物运价应考虑哪些因素?\end{CJK}
\\ \textbf{答:}  ①运输供求关系②竞争是影响水路货物运价的一个重要因素③国家对货物运价的调整④应考虑与其他运输部门之间的货运比价关系⑤货物对运费的负担能力.
\item \begin{CJK}{UTF8}{you} ISO9000族是什么意思?共有哪些版本标准?应用范围?\end{CJK}
\\ \textbf{答:} ISO9000是一项国际标准体系
\\  ISO9001用于有形业, ISO9002用于无形业, ISO9003用于环保, ISO9004用于要素指南.
\\
\end{enumerate}





\end{document}
