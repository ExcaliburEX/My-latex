
\documentclass[UTF8]{ctexart}
\usepackage{fancyhdr}
\usepackage{geometry}
\usepackage{amssymb}
\usepackage[fleqn]{amsmath}
\usepackage{enumerate}
\usepackage{CJK}
\usepackage{indentfirst}

\pagestyle{fancy}
\lhead{asdasd}
\chead{dsads}
\rhead{The performance of new graduates}
\lfoot{明星眼镜}
\cfoot{\hspace{-1.5in}地址:开发区广贤路38号(天星湖中学对面)}
\rfoot{电话:15251319693QQ:852046329}
\renewcommand{\headrulewidth}{2pt}
\renewcommand{\footrulewidth}{0.4pt}

\date{}
\setlength{\parindent}{2em}
\geometry{left=1.27cm,right=1.27cm,top=0pt,bottom=1.27cm}
\title{\textbf{\small 高等数学——定积分及其应用 \\  \emph{数学协会}\\$2016.12.10$}}
%\author{\vspace{0pt} \textbf{\small 数学协会}}
\begin{document}
\maketitle
\thispagestyle{fancy}
\vspace{-20mm} \textbf{\small 写在前面的话:前面我们说过对于不定积分的理解是函数的集合,与之相对的,这里我们强调定积分的实质意义是一块区域的面积,是对一块不规则图形用无穷细分的方法求其面积的重要工具。这一章我们要掌握其性质,计算方法,广义积分以及它的具体的实际应用方法。}
\vspace{-8mm} \section{定积分}
\vspace{-4mm} \subsection{概念}  \vspace{-2mm} \small 定义
\\ $\displaystyle{\int _a^b f(x)dx =\lim_{\lambda \to 0} \sum_{i=1}^n f(\xi_i)\Delta x_i}$
\\可积条件:(1)若$f(x)$在$[a,b]$上连续,则$f(x)$在$[a,b]$上可积.
\\(2)若$f(x)$是在$[a,b]$上只有有限个间断点的有界函数,则$f(x)$在$[a,b]$上可积.
\\例1 利用定积分求极限:$\displaystyle{\lim_{n \to \infty} (\frac{1}{\sqrt{4n^2-1^2}}+\frac{1}{\sqrt{4n^2-2^2}}+\cdot\cdot\cdot+\frac{1}{\sqrt{4n^2-n^2}})}$
\\解:$\displaystyle{f(\xi_i)=\frac{1}{\sqrt{4-\xi_i}}=\frac{1}{\sqrt{4-(\frac{i}{n})^2}},\Delta x_i=\frac{1}{n}}$
\\则$\displaystyle{\lim_{n \to \infty} \sum_{i=1}^n \frac{1}{\sqrt{4n^2-i^2}}=\lim_{n \to \infty} \frac{1}{\sqrt{4-(\frac{i}{n})^2}}\cdot\frac{1}{n}=\lim_{n \to \infty} \frac{1}{\sqrt{4-\xi^2}}\cdot\Delta x_i}$
\\=$\displaystyle{\int_0^1 \frac{1}{\sqrt{4-x^2}}dx}$
\vspace{-4mm} \subsection{定积分中值定理} \vspace{-2mm}\small 若$f(x)$在$[a,b]$上连续,则在$[a,b]$上至少存在一点$\xi$,使得$$\displaystyle{\int _a^b f(x)dx=f(\xi)(b-a)}$$
\\例2 设$f(x)$在$[0,1]$上连续,且$f(x)$在$(0,1)$上可导,又$3\displaystyle{\int_{\frac{2}{3}}^1 f(x)dx=f(0)}$,证明:至少存在一点$\xi\in (0,1)$,使$f'(\xi)=0$.(结合定积分中值定理与罗尔定理证明)
%\begin{CJK}{UTF8}{fs}
%\begin{CJK}{UTF8}{fs}

\vspace{-4mm} \subsection{变上限积分} \vspace{-2mm} 求导法则($\displaystyle{\int_a^x f(t)dt }$是$[a,b]$上连续函数$f(x)$的一个原函数.)
\\ (1)$\displaystyle{\frac{d}{dx} \int _a^x f(t)dt=f(x)}$;
\\(2)$\displaystyle{\frac{d}{dx}\int_a^{\beta(x)} f(t)dt=f[\beta(x)]\beta'(x) }$;
\\ (3)$\displaystyle{\frac{d}{dx}\int_{\alpha(x)}^{\beta(x)} f(t)dt=f[\beta(x)]\beta'(x)-f[\alpha(x)]\alpha'(x)} $
\\ 例3 设函数$f(x)$连续,且$f(0)\neq 0$,求极限:$\displaystyle{\lim_{x \to 0} \frac{\int _0^x (x-t)f(t)dt}{x\int_0^x f(x-t)dt}.}$
\\解:解决此类问题一般需要用洛必达法则.
\\(令$x-t=u$)因$\displaystyle{\int_0^x f(t)dt=\int_x^0 f(u)(-du)=\int_0^x f(u)du.}$
\\原式=$\displaystyle{\lim_{x \to 0} \frac{x\int_0^xf(t)dt-\int_0^xtf(t)dt}{x\int_0^xf(u)du}}$
\\$=\displaystyle{\frac{\int_0^x f(t)dt+xf(x)-xf(x)}{\int_0^x f(u)du+xf(x)}=\lim_{x \to 0} \frac{\int_0^x f(t)dt}{\int _0^x f(u)du+xf(x)}\xrightarrow{\mbox{积分中值定理}}\lim_{x \to 0}\frac{xf(\xi)}{xf(\xi)+xf(x)}=\frac{f(0)}{f(0)+f(0)}}$
\\$\displaystyle{=\frac{1}{2}}$(因$x\rightarrow 0$时,$\xi\rightarrow 0$,且$f(x)$连续.)
\vspace{-4mm} \subsection{定积分计算方法} \vspace{-2mm} $Newton-Leibniz$公式
\\$$\displaystyle{\int_a^b f(x)dx=F(b)-F(a)} $$
除此以外,定积分的换元积分法与分部积分法与不定积分区别不大,我们着重强调奇偶性与周期性.
\\ 奇偶性:设$f(x)$在$[-a,a]$上连续,若$f(x)$是奇函数,则$\displaystyle{\int_{-a}^a f(x)dx=0} $.若$f(x)$是偶函数,则$\displaystyle{\int_{-a}^a f(x)dx=2\int_0^a f(x)dx} $
\\ 例如 $\displaystyle{\int_{-1}^1 x^3e^{-x^2}=0 }$(奇乘偶为奇),$\displaystyle{\int_{-\frac{\pi}{2}}^{\frac{\pi}{2}} sin^4x dx=2\int _0^\frac{\pi}{2} sin^4xdx}$(之后可以利用$\displaystyle{\int _0^\frac{\pi}{2} sin^ndx}$的公式求解.)
\\ 周期性:设$f(x)$是定义在$(-\infty,+\infty)$上以$T$为周期的周期函数,$a$为任意常数,则$\displaystyle{\int_a^{a+T} f(x)dx=\int _0^T f(x)dx}$.
\\ 例4 $\displaystyle{\int _0^{100\pi} \sqrt{1-cos2x}dx=\int_0^{100\pi} \sqrt{2sin^2x}dx=\sqrt{2}\int_0^{100\pi}|sinx|dx}$
\\ $f(x)=|sinx|$是以$\pi$为周期的周期函数.
\\ 原式=$\displaystyle{\sqrt{2}(\int_0^{\pi}|sinx|dx+\int_\pi^{2\pi}|sinx|dx)+\cdot\cdot\cdot+\int_{99\pi}^{100\pi}|sinx|dx}$
\\$=\displaystyle{100\sqrt{2} \int_0^\pi |sinx|dx}$
\\$=\displaystyle{100\sqrt{2}\int_0^\pi sinxdx}$
\\$=\displaystyle{100\sqrt{2}(-cosx)|_0^\pi=200\sqrt{2}}$
\\ 常用积分公式:
\\ (1)$\displaystyle{\int_{-a}^a f(x)dx=\int _0^a [f(x)+f(-x)]dx}.$
\\(2)$\displaystyle{\int_0^\pi xf(sinx)dx=\frac{\pi}{2}\int_0^\pi f(sinx)dx.}$
\\(3)$\displaystyle{\int _0^{\frac{\pi}{2}} f(sinx)dx=\int _0^{\frac{\pi}{2}} f(cosx)dx.}$
\\(4)$\displaystyle{\int _0^{\frac{\pi}{2}} sin^nxdx =\int _0^{\frac{\pi}{2}} cos^nxdx=\left\{
\begin {array}{rcl}
\frac{n-1}{n}\cdot\frac{n-3}{n-2}\cdot\cdot\cdot\frac{3}{4}\cdot\frac{1}{2}\cdot\frac{\pi}{2},\mbox{n为正偶数}\\
\frac{n-1}{n}\cdot\frac{n-3}{n-2}\cdot\cdot\cdot\frac{4}{5}\cdot\frac{2}{3}\cdot1,\mbox{n为大于1的奇数.}
\end{array}\right.}$



\vspace{-4mm} \subsection{广义积分}  \vspace{-2mm} \small (1)无穷限的广义积分.
\\若$f(x)$在积分区间连续,则
$$\displaystyle{\int_a^{+\infty}f(x)dx=\lim_{b \to +\infty} \int_a^b f(x)dx(a<b);}$$
$$\displaystyle{\int_{-\infty}^b f(x)dx=\lim_{a \to -\infty} \int_a^b f(x)dx(a<b);}$$
$$\displaystyle{\int_{-\infty}^{+\infty}f(x)dx=\lim_{a \to -\infty} \int _a^0 f(x)dx+\lim_{b \to +\infty} \int _0^b f(x)dx}.$$
(2)无界函数的广义积分.
\\若$f(x)$在$(a,b]$上连续,在点$a$的右邻域内无界,则
$$\displaystyle{\int_a^b f(x)dx=\lim_{\epsilon \to 0^+}\int_{a+\epsilon}^b f(x)dx(0<\epsilon<b-a)}.$$
若$f(x)$在$[a,b)$上连续,在点$b$的左邻域内无界,则
$$\displaystyle{\int_a^b f(x)dx=\lim_{\epsilon \to 0^+}\int_{a}^{b-\epsilon} f(x)dx(0<\epsilon<b-a)}.$$
若$f(x)$在$[a,c)$及$(c,b]$上连续,在点$c$的邻域内无界,则
$$\displaystyle{\int_a^b f(x)dx=\lim_{\epsilon \to 0^+}\int_{a}^{c-\epsilon} f(x)dx+\lim_{\epsilon' \to 0^+}\int_{c+\epsilon'}^b f(x)dx(0<\epsilon<c-a,0<\epsilon'<b-c)}.$$
以上定义中,若极限存在,则广义积分收敛,否则称其发散.
\\例5 计算$\displaystyle{\int_{-\infty}^{+\infty} \frac{1}{1+x^2}dx.}$
\\解:$\displaystyle{\int_{-\infty}^{+\infty} \frac{1}{1+x^2}dx=\int_{-\infty}^0 \frac{1}{1+x^2}dx+\int_0^{+\infty} \frac{1}{1+x^2}dx}$
\\$=\displaystyle{\lim_{a \to -\infty}\int_a^0 \frac{dx}{1+x^2}+\lim_{b \to +\infty}\int_0^b \frac{dx}{1+x^2}}$
\\$=\displaystyle{\lim_{a \to -\infty} arctan|_0^a+\lim_{b \to +\infty} arctan|_0^b}$
\\$=\displaystyle{-(-\frac{\pi}{2})+\frac{\pi}{2}}$
\\$=\pi.$
\\不能写成$\displaystyle{\int_{-\infty}^{+\infty} \frac{1}{1+x^2}dx=\lim_{a \to +\infty} \int_{-a}^a \frac{1}{1+x^2}dx}.$
\\若$\displaystyle{\lim_{a \to +\infty} \int_{-a}^a \frac{1}{1+x^2}dx}$存在,不能保证收敛,称此极限值为$\displaystyle{\int_{-\infty}^{+\infty} \frac{1}{1+x^2}dx}$的柯西主值,记为:
$$P.V.\displaystyle{\int_{-\infty}^{+\infty} \frac{1}{1+x^2}dx=\lim_{a \to +\infty} \int_{-a}^a \frac{1}{1+x^2}dx}.$$
\\$\varGamma$——函数
\\广义积分$\displaystyle{\varGamma(p)=\int _0^{+\infty} x^{p-1} e^{-x}dx}$,当$p>0$时,其收敛.
\\性质:(1)$\displaystyle{\varGamma(1)=1}$
\\(2)$\displaystyle{\varGamma(p+1)=p\varGamma(p)}$
\\(3)$\displaystyle{\varGamma(n+1)=n!(n\in N)}$
\vspace{-8mm} \section{几何物理应用}几何应用
\\1.平面图形面积
\\(1)直角坐标:$S=\displaystyle{\int_a^b |g(x)-f(x)|dx.}$
\\(2)极坐标:$S=\displaystyle{\frac{1}{2}\int_\alpha^\beta |r_2^2(\theta)-r_1^2(\theta)|d\theta}.$
\\2.体积
\\(1)平行截面面积的立体的体积:$\displaystyle{V=\int_a^b A(x)dx.}$
\\(2)旋转体体积:$\displaystyle{V=\int_a^b \pi f^2(x)dx.}$
\\3.平面曲线的弧长
\\(1)直角坐标:$\displaystyle{s=\int_a^b \sqrt{1+f'^2(x)dx}.}$
\\(2)参数方程:$\displaystyle{s=\int_\alpha^\beta \sqrt{\varphi'^2(t)+\psi'^2(t)}dt}.$
\\(3)极坐标:$\displaystyle{s=\int_\alpha^\beta \sqrt{r^2(\theta)+r'^2(\theta)}d\theta}$
\\物理应用
\\1.变力沿直线做功:$\displaystyle{\int_a^b F(x)dx.}$
\\2.液体的侧压力:$\displaystyle{\int_a^b \gamma xf(x)dx.}$($\gamma$为液体密度.)
\\函数值的平均值
\\$\displaystyle{\tilde{y}=\frac{1}{b-a} \int_a^b f(x)dx.}$

\centerline{\textbf{本章练习}}

\begin{enumerate}[(1)]
\item $\displaystyle{\int_1^{\sqrt{3}} \frac{dx}{x^2\sqrt{1+x^2} } }$
\item $\displaystyle{\int_{-\frac{\pi}{2}}^\frac{\pi}{2} 4cos^4\theta d\theta} $
\item 证明$\displaystyle{\int _x^1 \frac{dt}{1+t^2}=\int_1^{\frac{1}{x}} \frac{dt}{1+t^2}.}$
\item $\displaystyle{\int_0^{+\infty} e^{-pt}sin\omega tdt(p>0,\omega>0)}$


\item $\displaystyle{\lim_{n\to \infty} \frac{1^p+2^p+\cdot\cdot\cdot+n^p}{n^{p+1}}(p>0})$
\item $\displaystyle{\int_0^{\frac{\pi}{2}} \sqrt{1-sin2x}dx}$
\item $\displaystyle{\int_0^{+\infty} \frac{dx}{1+x^4}}.$
\item 设函数$f(x)$在$[0,1]$上连续,在$(0,1)$内可导,且$2\displaystyle{\int_0^{\frac{1}{2}} xf(x)dx=f(1).}$证明:在$(0,1)$内至少存在一点$\xi$,使$f'(\xi)=-\displaystyle{\frac{f(\xi)}{\xi}}$
\item 求由摆线$x=a(t-sint)$,$y=a(1-cost)$的一拱$(0\leq t\leq 2\pi)$与横轴所围成的图形的面积.
\item 把星形线$\displaystyle{x^{\frac{2}{3}}+y^{\frac{2}{3}}=a^{\frac{2}{3}}}$所围成的图形绕x轴旋转,计算所得旋转体的体积.
\item 求心形线$\rho =a(1+cos\theta)$的全长.
\item 求抛物线$y=\dfrac{1}{2}x^2$被圆$x^2+y^2=3$所截下的有限部分的弧长.

\end{enumerate}
\end{document}
